\documentclass[14pt, a4paper]{extarticle}
\usepackage{GOST}
\usepackage{array}
\usepackage{verbatim}
\usepackage[detect-all]{siunitx}
\usepackage{amsmath}
\usepackage{amssymb}
\usepackage[utf8]{inputenc}
\usepackage{hyperref}

\usepackage{ifthen}


\usepackage{tempora}


\makeatletter
\renewcommand\@biblabel[1]{#1.}
\makeatother

% Для листинга кода:
\usepackage{listings}
\lstset{ %
	language=python,                 % выбор языка для подсветки (здесь это С)
	basicstyle=\small\sffamily, % размер и начертание шрифта для подсветки кода
	numbers=left,               % где поставить нумерацию строк (слева\справа)
	numberstyle=\tiny,           % размер шрифта для номеров строк
	stepnumber=1,                   % размер шага между двумя номерами строк
	numbersep=5pt,                % как далеко отстоят номера строк от подсвечиваемого кода
	showspaces=false,            % показывать или нет пробелы специальными отступами
	showstringspaces=false,      % показывать или нет пробелы в строках
	showtabs=false,             % показывать или нет табуляцию в строках
	frame=single,              % рисовать рамку вокруг кода
	tabsize=2,                 % размер табуляции по умолчанию равен 2 пробелам
	captionpos=t,              % позиция заголовка вверху [t] или внизу [b] 
	breaklines=true,           % автоматически переносить строки (да\нет)
	breakatwhitespace=false, % переносить строки только если есть пробел
	escapeinside={\#*}{*)}   % если нужно добавить комментарии в коде
}


%для графиков
\usepackage{pgfplots}
\usepackage{filecontents}
\usetikzlibrary{datavisualization}
\usetikzlibrary{datavisualization.formats.functions}

\begin{document}
	
	\begin{table}[ht]
		\centering
		\begin{tabular}{|c|p{400pt}|} 
			\hline
			\begin{tabular}[c]{@{}c@{}} \includegraphics[scale=1]{baum.jpg} \\\end{tabular} &
			\footnotesize\begin{tabular}[c]{@{}c@{}}\textbf{Министерство~науки~и~высшего~образования~Российской~Федерации}\\\textbf{Федеральное~государственное~бюджетное~образовательное~учреждение}\\\textbf{~высшего~образования}\\\textbf{«Московский~государственный~технический~университет}\\\textbf{имени~Н.Э.~Баумана}\\\textbf{(национальный~исследовательский~университет)»}\\\textbf{(МГТУ~им.~Н.Э.~Баумана)}\\\end{tabular}  \\
			\hline
		\end{tabular}
	\end{table}
	\noindent\rule{\textwidth}{4pt}
	\noindent\rule[14pt]{\textwidth}{1pt}
	\hfill 
	\noindent
	\makebox{ФАКУЛЬТЕТ~}%
	\makebox[\textwidth][l]{\underline{~«Информатика и системы управления»~~~~~~~~~~~~~~~~~~~~~~~~~~~~~~~~~}}%
	\\
	\noindent
	\makebox{КАФЕДРА~}%
	\makebox[\textwidth][l]{\underline{~«Программное обеспечение ЭВМ и информационные технологии»~}}%
	
	
	\begin{center}
		\vspace{1.5cm}
		{\bf\huge Отчёт\par}
		{\bf\Large по лабораторной работе № 7\par}
		\vspace{0.7cm}
	\end{center}
	
	
	\noindent
	\makebox{\large{\bf Название:}~~~}
	\makebox[\textwidth][l]{\large\underline{Определение вероятности отказа при помощи GPSS~~}}
	
	\noindent
	\makebox{\large{\bf Дисциплина:}~~~}
	\makebox[\textwidth][l]{\large\underline{~Моделирование~~~~~~~~~~~~~~~~~~~~~~~~~~}}\\
	
	\vspace{1.5cm}
	\noindent
	\begin{tabular}{l c c c c c}
		Студент      & ~ИУ7-75Б~               & \hspace{2.5cm} & \hspace{2cm}                 & &  Д.В. 
		Сусликов \\\cline{2-2}\cline{4-4} \cline{6-6} 
		\hspace{3cm} & {\footnotesize(Группа)} &                & {\footnotesize(Подпись, дата)} & & {\footnotesize(И.О. Фамилия)}
	\end{tabular}
	
	\noindent
	\begin{tabular}{l c c c c}
		Преподаватель & \hspace{5cm}   & \hspace{2cm}                 & & ~~~~~~И.В. Рудаков~~~~~~\\\cline{3-3} \cline{5-5} 
		\hspace{3cm}  &                & {\footnotesize(Подпись, дата)} & & {\footnotesize(И.О. Фамилия)}
	\end{tabular}
	
	\vspace{0.6cm}
	\begin{center}	
		\vfill
		\large \textit {Москва, 2021}
	\end{center}
	
	\thispagestyle {empty}
	\pagebreak
	
	% СОДЕРЖАНИЕ 
	\clearpage
	\tableofcontents
		
	% ВВЕДЕНИЕ
	\clearpage
	\section*{Задание}
	\addcontentsline{toc}{section}{Задание}
	В информационный центр приходят клиенты через интервал времени 10 +- 2 минуты. Если все три имеющихся оператора заняты, клиенту отказывают в обслуживании. Операторы имеют разную производительность и могут обеспечивать обслуживание среднего запроса пользователя за 20 +- 5; 40 +- 10; 40 +- 20. Клиенты стремятся занять свободного оператора с максимальной производительностью. Полученные запросы сдаются в накопитель. Откуда выбираются на обработку. На первый компьютер запросы от 1 и 2-ого операторов, на второй – запросы от 3-его. Время обработки запросов первым и 2-м компьютером равны соответственно 15 и 30 мин. Промоделировать процесс обработки 300 запросов. 
	Для выполнения поставленного задания необходимо создать концептуальную модель в терминах СМО, определить эндогенные и экзогенные переменные и уравнения модели. За единицу системного времени выбрать 0,01 минуты.

	
	\begin{figure}[h!]
		\centering{\includegraphics[scale=0.4]{source/task1.png}}
	\end{figure}
	
	\clearpage
	\section*{Аналитическая часть}
	\addcontentsline{toc}{section}{Аналитическая часть}
	В процессе взаимодействия клиентов с информационным центром возможно:
	
	1) Режим нормального обслуживания, т.е. клиент выбирает одного из свободных операторов, отдавая предпочтение тому у которого меньше номер.
	
	2) Режим отказа в обслуживании клиента, когда все операторы заняты. 
	\textbf{Переменные и уравнения имитационной модели }\par
	Эндогенные переменные: время обработки задания i-ым оператором, время решения этого задания j-ым компьютером.
	Экзогенные переменные: число обслуженных клиентов и число клиентов, получивших отказ.
	
	\begin{figure}[h!]
		\centering{\includegraphics[scale=0.5]{source/task2.png}}
	\end{figure}
	
	\begin{equation*}
		P_{\text{отк}} = \frac{C_{\text{отк}}}{C_{\text{отк}} + C_{\text{обсл}}}
	\end{equation*}
	
	\clearpage
	\section*{Результаты работы}
	\addcontentsline{toc}{section}{Результаты работы}
	Ниже представлены результаты работы программы.
	\begin{figure}[h!]
		\centering{\includegraphics[scale=0.9]{source/3}}
		\centering\caption{Результаты работы}
	\end{figure}
	
	\section*{Вывод}
	\addcontentsline{toc}{section}{Вывод}
	Процент потерь равен 23\%.
	
	
	\clearpage
	\section*{Листинги}
	\addcontentsline{toc}{section}{Листинги}
	
	\begin{figure}[h!]
		\centering{\includegraphics[scale=0.8]{source/listning1.png}}
	\end{figure}
	\newpage
	\begin{figure}[h!]
		\centering{\includegraphics[scale=0.8]{source/listning2.png}}
	\end{figure}
\end{document}